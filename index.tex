% Options for packages loaded elsewhere
\PassOptionsToPackage{unicode}{hyperref}
\PassOptionsToPackage{hyphens}{url}
\PassOptionsToPackage{dvipsnames,svgnames,x11names}{xcolor}
%
\documentclass[
  letterpaper,
  DIV=11,
  numbers=noendperiod]{scrreprt}

\usepackage{amsmath,amssymb}
\usepackage{iftex}
\ifPDFTeX
  \usepackage[T1]{fontenc}
  \usepackage[utf8]{inputenc}
  \usepackage{textcomp} % provide euro and other symbols
\else % if luatex or xetex
  \usepackage{unicode-math}
  \defaultfontfeatures{Scale=MatchLowercase}
  \defaultfontfeatures[\rmfamily]{Ligatures=TeX,Scale=1}
\fi
\usepackage{lmodern}
\ifPDFTeX\else  
    % xetex/luatex font selection
\fi
% Use upquote if available, for straight quotes in verbatim environments
\IfFileExists{upquote.sty}{\usepackage{upquote}}{}
\IfFileExists{microtype.sty}{% use microtype if available
  \usepackage[]{microtype}
  \UseMicrotypeSet[protrusion]{basicmath} % disable protrusion for tt fonts
}{}
\makeatletter
\@ifundefined{KOMAClassName}{% if non-KOMA class
  \IfFileExists{parskip.sty}{%
    \usepackage{parskip}
  }{% else
    \setlength{\parindent}{0pt}
    \setlength{\parskip}{6pt plus 2pt minus 1pt}}
}{% if KOMA class
  \KOMAoptions{parskip=half}}
\makeatother
\usepackage{xcolor}
\setlength{\emergencystretch}{3em} % prevent overfull lines
\setcounter{secnumdepth}{5}
% Make \paragraph and \subparagraph free-standing
\ifx\paragraph\undefined\else
  \let\oldparagraph\paragraph
  \renewcommand{\paragraph}[1]{\oldparagraph{#1}\mbox{}}
\fi
\ifx\subparagraph\undefined\else
  \let\oldsubparagraph\subparagraph
  \renewcommand{\subparagraph}[1]{\oldsubparagraph{#1}\mbox{}}
\fi


\providecommand{\tightlist}{%
  \setlength{\itemsep}{0pt}\setlength{\parskip}{0pt}}\usepackage{longtable,booktabs,array}
\usepackage{calc} % for calculating minipage widths
% Correct order of tables after \paragraph or \subparagraph
\usepackage{etoolbox}
\makeatletter
\patchcmd\longtable{\par}{\if@noskipsec\mbox{}\fi\par}{}{}
\makeatother
% Allow footnotes in longtable head/foot
\IfFileExists{footnotehyper.sty}{\usepackage{footnotehyper}}{\usepackage{footnote}}
\makesavenoteenv{longtable}
\usepackage{graphicx}
\makeatletter
\def\maxwidth{\ifdim\Gin@nat@width>\linewidth\linewidth\else\Gin@nat@width\fi}
\def\maxheight{\ifdim\Gin@nat@height>\textheight\textheight\else\Gin@nat@height\fi}
\makeatother
% Scale images if necessary, so that they will not overflow the page
% margins by default, and it is still possible to overwrite the defaults
% using explicit options in \includegraphics[width, height, ...]{}
\setkeys{Gin}{width=\maxwidth,height=\maxheight,keepaspectratio}
% Set default figure placement to htbp
\makeatletter
\def\fps@figure{htbp}
\makeatother

\KOMAoption{captions}{tableheading}
\makeatletter
\makeatother
\makeatletter
\@ifpackageloaded{bookmark}{}{\usepackage{bookmark}}
\makeatother
\makeatletter
\@ifpackageloaded{caption}{}{\usepackage{caption}}
\AtBeginDocument{%
\ifdefined\contentsname
  \renewcommand*\contentsname{Table of contents}
\else
  \newcommand\contentsname{Table of contents}
\fi
\ifdefined\listfigurename
  \renewcommand*\listfigurename{List of Figures}
\else
  \newcommand\listfigurename{List of Figures}
\fi
\ifdefined\listtablename
  \renewcommand*\listtablename{List of Tables}
\else
  \newcommand\listtablename{List of Tables}
\fi
\ifdefined\figurename
  \renewcommand*\figurename{Figure}
\else
  \newcommand\figurename{Figure}
\fi
\ifdefined\tablename
  \renewcommand*\tablename{Table}
\else
  \newcommand\tablename{Table}
\fi
}
\@ifpackageloaded{float}{}{\usepackage{float}}
\floatstyle{ruled}
\@ifundefined{c@chapter}{\newfloat{codelisting}{h}{lop}}{\newfloat{codelisting}{h}{lop}[chapter]}
\floatname{codelisting}{Listing}
\newcommand*\listoflistings{\listof{codelisting}{List of Listings}}
\makeatother
\makeatletter
\@ifpackageloaded{caption}{}{\usepackage{caption}}
\@ifpackageloaded{subcaption}{}{\usepackage{subcaption}}
\makeatother
\makeatletter
\@ifpackageloaded{tcolorbox}{}{\usepackage[skins,breakable]{tcolorbox}}
\makeatother
\makeatletter
\@ifundefined{shadecolor}{\definecolor{shadecolor}{rgb}{.97, .97, .97}}
\makeatother
\makeatletter
\makeatother
\makeatletter
\makeatother
\makeatletter
\@ifpackageloaded{fontawesome5}{}{\usepackage{fontawesome5}}
\makeatother
\ifLuaTeX
  \usepackage{selnolig}  % disable illegal ligatures
\fi
\IfFileExists{bookmark.sty}{\usepackage{bookmark}}{\usepackage{hyperref}}
\IfFileExists{xurl.sty}{\usepackage{xurl}}{} % add URL line breaks if available
\urlstyle{same} % disable monospaced font for URLs
\hypersetup{
  pdftitle={Psychology of Language},
  pdfauthor={Casey L. Roark},
  colorlinks=true,
  linkcolor={blue},
  filecolor={Maroon},
  citecolor={Blue},
  urlcolor={Blue},
  pdfcreator={LaTeX via pandoc}}

\title{Psychology of Language}
\author{Casey L. Roark}
\date{2023-06-08}

\begin{document}
\maketitle
\ifdefined\Shaded\renewenvironment{Shaded}{\begin{tcolorbox}[frame hidden, sharp corners, boxrule=0pt, borderline west={3pt}{0pt}{shadecolor}, breakable, interior hidden, enhanced]}{\end{tcolorbox}}\fi

\renewcommand*\contentsname{Table of contents}
{
\hypersetup{linkcolor=}
\setcounter{tocdepth}{2}
\tableofcontents
}
\bookmarksetup{startatroot}

\hypertarget{welcome}{%
\chapter*{Welcome!}\label{welcome}}
\addcontentsline{toc}{chapter}{Welcome!}

\markboth{Welcome!}{Welcome!}

\begin{figure}

{\centering \includegraphics[width=4.0625in,height=\textheight]{images/volodymyr-hryshchenko-V5vqWC9gyEU-unsplash.jpg}

}

\end{figure}

\hypertarget{learn-about-the-cognitive-and-neural-basis-of-our-remarkable-abilities-to-communicate}{%
\subsubsection*{Learn about the cognitive and neural basis of our
remarkable abilities to
communicate}\label{learn-about-the-cognitive-and-neural-basis-of-our-remarkable-abilities-to-communicate}}
\addcontentsline{toc}{subsubsection}{Learn about the cognitive and
neural basis of our remarkable abilities to communicate}

~

\hypertarget{psyc-712w---fall-2023}{%
\paragraph*{\texorpdfstring{{PSYC 712W - Fall
2023}}{PSYC 712W - Fall 2023}}\label{psyc-712w---fall-2023}}
\addcontentsline{toc}{paragraph}{{PSYC 712W - Fall 2023}}

\hypertarget{department-of-psychology}{%
\paragraph*{\texorpdfstring{{Department of
Psychology}}{Department of Psychology}}\label{department-of-psychology}}
\addcontentsline{toc}{paragraph}{{Department of Psychology}}

\hypertarget{university-of-new-hampshire}{%
\paragraph*{\texorpdfstring{{University of New
Hampshire}}{University of New Hampshire}}\label{university-of-new-hampshire}}
\addcontentsline{toc}{paragraph}{{University of New Hampshire}}

~

\hypertarget{instructor}{%
\subsection*{Instructor}\label{instructor}}
\addcontentsline{toc}{subsection}{Instructor}

\begin{itemize}
\tightlist
\item
  \faIcon{user} ~ \href{www.roarklab.com}{Dr.~Casey L. Roark}
\item
  \faIcon{university} ~ 440 McConnell Hall
\item
  \faIcon{envelope} ~
  \href{mailto:casey.roark@unh.edu}{\nolinkurl{casey.roark@unh.edu}}
\item
  \faIcon{twitter} ~
  \href{https://www.twitter.com/caseyroark}{caseyroark}
\item
  \faIcon{calendar-check} ~ Wednesday 1-3 pm or by appointment
\end{itemize}

\hypertarget{course-details}{%
\subsection*{Course details}\label{course-details}}
\addcontentsline{toc}{subsection}{Course details}

\begin{itemize}
\tightlist
\item
  \faIcon{calendar} ~ Tuesday and Thursday
\item
  \faIcon{calendar-alt} ~ Fall 2023
\item
  \faIcon{clock} ~ 2:10-3:30 pm
\item
  \faIcon{location-dot} ~ HORT 215
\end{itemize}

\hypertarget{contacting-me}{%
\subsection*{Contacting me}\label{contacting-me}}
\addcontentsline{toc}{subsection}{Contacting me}

E-mail is the best way to get in contact with me. I will try to respond
to all course-related e-mails within 24 hours (really), but also
remember that life can be busy and chaotic for everyone (including me!),
so if I don't respond right away, don't worry!

\bookmarksetup{startatroot}

\hypertarget{syllabus}{%
\chapter*{Syllabus}\label{syllabus}}
\addcontentsline{toc}{chapter}{Syllabus}

\markboth{Syllabus}{Syllabus}

\hypertarget{course-information}{%
\section*{Course information}\label{course-information}}
\addcontentsline{toc}{section}{Course information}

\markright{Course information}

\hypertarget{instructor-1}{%
\subsection*{Instructor}\label{instructor-1}}
\addcontentsline{toc}{subsection}{Instructor}

\begin{itemize}
\tightlist
\item
  \faIcon{user} ~ \href{www.roarklab.com}{Dr.~Casey L. Roark}
\item
  \faIcon{university} ~ 440 McConnell Hall
\item
  \faIcon{envelope} ~
  \href{mailto:casey.roark@unh.edu}{\nolinkurl{casey.roark@unh.edu}}
\item
  \faIcon{twitter} ~
  \href{https://www.twitter.com/caseyroark}{caseyroark}
\item
  \faIcon{calendar-check} ~ Wednesday 1-3 pm or by appointment
\end{itemize}

\hypertarget{course-details-1}{%
\subsection*{Course details}\label{course-details-1}}
\addcontentsline{toc}{subsection}{Course details}

\begin{itemize}
\tightlist
\item
  \faIcon{caret-right} ~ PSYC 712W -- Psychology of Language
\item
  \faIcon{check-square} ~ Pre-requisites: PSYC 402, 505, 512, 513 or
  permission
\item
  \faIcon{calendar} ~ Tuesday and Thursday
\item
  \faIcon{calendar-alt} ~ Fall 2023
\item
  \faIcon{clock} ~ 2:10-3:30 pm
\item
  \faIcon{location-dot} ~ HORT 215
\end{itemize}

\hypertarget{course-overview}{%
\section*{Course Overview}\label{course-overview}}
\addcontentsline{toc}{section}{Course Overview}

\markright{Course Overview}

Psychology of Language explores the cognitive and neural bases of human
language. We use language in our everyday lives mostly effortlessly and
without thinking. But underneath it all, language is extraordinarily
complex. In this course, through lectures, reading, writing, and
discussion, we will explore some of the feats and challenges of human
language including (but not limited to):

\begin{itemize}
\tightlist
\item
  Components of language such as speech perception, speech production,
  reading, writing
\item
  Whether language is a uniquely human and innate ability
\item
  How children develop the ability to speak and understand language
\item
  How language functions in the brain
\item
  The relationship between language and thought
\item
  What happens when language does not function typically
\end{itemize}

\hypertarget{course-materials-required}{%
\section*{Course materials (required)}\label{course-materials-required}}
\addcontentsline{toc}{section}{Course materials (required)}

\markright{Course materials (required)}

\begin{itemize}
\tightlist
\item
  Christiansen, M. H. \& Chater, N. (2022).
  \href{https://bookshop.org/p/books/the-language-game-how-improvisation-created-language-and-changed-the-world-nick-chater/16984145?ean=9781541674981}{The
  Language Game: How improvisation created language and changed the
  world}. Basic Books.
\item
  Sedivy, J. (2014).
  \href{https://www.amazon.com/Language-Mind-Psycholinguistics-Julie-Sedivy/dp/1605357057}{Language
  in Mind: An introduction to psycholinguistics}. Second edition. Oxford
  University Press.
\item
  Other PDFs will be provided on the
  \href{https://my.unh.edu/canvas}{course website}.
\end{itemize}

\hypertarget{course-description}{%
\section*{Course Description}\label{course-description}}
\addcontentsline{toc}{section}{Course Description}

\markright{Course Description}

Theories of language structure, functions of human language, meaning,
relationship of language to other mental processes, language
acquisition, indices of language development, speech perception,
reading.

\hypertarget{course-learning-objectives}{%
\section*{Course Learning Objectives}\label{course-learning-objectives}}
\addcontentsline{toc}{section}{Course Learning Objectives}

\markright{Course Learning Objectives}

Upon completion of this course students will

\begin{itemize}
\tightlist
\item
\item
\end{itemize}

\hypertarget{course-structure}{%
\section*{Course Structure}\label{course-structure}}
\addcontentsline{toc}{section}{Course Structure}

\markright{Course Structure}

\href{https://my.unh.edu/canvas}{myCourses} is the learning management
tool we use for this course. The course is organized by class meetings.
You will be able to find external readings and submit assignments.
Please do not message me through myCourses -- email me instead.

\hypertarget{grades}{%
\section*{Grades}\label{grades}}
\addcontentsline{toc}{section}{Grades}

\markright{Grades}

\begin{longtable}[]{@{}
  >{\raggedright\arraybackslash}p{(\columnwidth - 4\tabcolsep) * \real{0.3261}}
  >{\raggedright\arraybackslash}p{(\columnwidth - 4\tabcolsep) * \real{0.1522}}
  >{\raggedright\arraybackslash}p{(\columnwidth - 4\tabcolsep) * \real{0.5217}}@{}}
\toprule\noalign{}
\begin{minipage}[b]{\linewidth}\raggedright
Item
\end{minipage} & \begin{minipage}[b]{\linewidth}\raggedright
\% of final grade
\end{minipage} & \begin{minipage}[b]{\linewidth}\raggedright
Requirements
\end{minipage} \\
\midrule\noalign{}
\endhead
\bottomrule\noalign{}
\endlastfoot
Quizzes & 15\% & There are 5 quizzes for this course, which are
scheduled on the course calendar. Each quiz is worth 3\% of the final
grade (5 quizzes x 3\% = 15\% total) \\
Class Attendance, Participation, and Discussion & 20\% & Attending and
participating in class are important parts of this course. Attendance
will be taken in each class period. Participation will be assessed for
each class period and students will be given multiple options for
participation over the course of the semester (e.g., group discussion,
pair discussion, self-reflection, in class activities, etc.) \\
Thought Papers & 20\% & There are four required thought papers for this
course. Each week will come with the opportunity to complete a thought
paper and students are expected to complete at least four over the
course of the semester. Each paper will be worth 5\% of the final grade
(4 papers x 5\% = 20\% total) \\
Lab Reports & 15\% & Students will be required to complete three lab
reports during the semester. Each lab report will be worth 5\% of the
final grade (3 lab reports x 5\% = 15\% total) \\
Final Presentation & 15\% & There will be a final presentation in the
final weeks of the semester. \\
Final Paper & 15\% & There will be a final paper along with the
presentation in the final weeks of the semester. \\
\end{longtable}

\hypertarget{technical-requirements-and-technical-support}{%
\section*{Technical Requirements and Technical
Support}\label{technical-requirements-and-technical-support}}
\addcontentsline{toc}{section}{Technical Requirements and Technical
Support}

\markright{Technical Requirements and Technical Support}

See \href{https://online.unh.edu/technical-requirements}{website
listings} for current recommendations and requirements related to this
course. For technical assistance please call (603) 862-4242 or fill out
an \href{https://itsupport.unh.edu/onlinelearning/}{online support
form}.

\hypertarget{university-disability-accommodations}{%
\section*{University Disability
Accommodations}\label{university-disability-accommodations}}
\addcontentsline{toc}{section}{University Disability Accommodations}

\markright{University Disability Accommodations}

The University is committed to providing students with documented
disabilities equal access to all university programs and facilities. If
you think you have a disability requiring accommodations, you must
register with \href{http://www.unh.edu/studentaccessibility}{Student
Accessibility Services (SAS)} or directly contact SAS at (603)
862-2607.\\
Please provide me with that information privately so that we can review
those accommodations.

\hypertarget{academic-honesty}{%
\section*{Academic Honesty}\label{academic-honesty}}
\addcontentsline{toc}{section}{Academic Honesty}

\markright{Academic Honesty}

Students are required to abide by the UNH Academic Honesty policy
located in the \href{https://catalog.unh.edu/srrr/}{Student Rights,
Rules, and Responsibilities Handbook}.

Plagiarism of any type may be grounds for receiving an ``F'' in an
assignment or an ``F'' in the overall course. Plagiarism is defined as
``the unattributed use of the ideas, evidence, or words of another
person, or the conveying the false impression that the arguments and
writing in a paper are your own.'' (UNH Academic Honesty Policy, 09.3)
Incidents are reported to the school dean and may be grounds for further
action. If you have questions about proper citation refer to your
department's writing guidelines. You can contact me at any time on this
issue. Additional resources can be found through the
\href{http://libraryguides.unh.edu/unhmcitingsources}{library guides on
citing sources}.

\hypertarget{note}{%
\section*{Note}\label{note}}
\addcontentsline{toc}{section}{Note}

\markright{Note}

This syllabus is subject to change. Students will be promptly notified
of any changes.

\bookmarksetup{startatroot}

\hypertarget{course-schedule}{%
\chapter*{Course Schedule}\label{course-schedule}}
\addcontentsline{toc}{chapter}{Course Schedule}

\markboth{Course Schedule}{Course Schedule}

\begin{longtable}[]{@{}
  >{\raggedright\arraybackslash}p{(\columnwidth - 8\tabcolsep) * \real{0.0750}}
  >{\raggedright\arraybackslash}p{(\columnwidth - 8\tabcolsep) * \real{0.1000}}
  >{\raggedright\arraybackslash}p{(\columnwidth - 8\tabcolsep) * \real{0.2125}}
  >{\raggedright\arraybackslash}p{(\columnwidth - 8\tabcolsep) * \real{0.3125}}
  >{\raggedright\arraybackslash}p{(\columnwidth - 8\tabcolsep) * \real{0.3000}}@{}}
\toprule\noalign{}
\begin{minipage}[b]{\linewidth}\raggedright
Week
\end{minipage} & \begin{minipage}[b]{\linewidth}\raggedright
Date
\end{minipage} & \begin{minipage}[b]{\linewidth}\raggedright
Topics
\end{minipage} & \begin{minipage}[b]{\linewidth}\raggedright
Readings
\end{minipage} & \begin{minipage}[b]{\linewidth}\raggedright
Assignments
\end{minipage} \\
\midrule\noalign{}
\endhead
\bottomrule\noalign{}
\endlastfoot
1 & T 8/29 & Introduction to Psychology of Language & & \\
1 & Th 8/31 & Science of Language & Sedivy - chapter 1 C\&C - chapters
1-2 & \\
2 & T 9/5 & Brief Introduction to Language in the Brain & Sedivy -
chapter 3 & \\
2 & Th 9/7 & Speech Perception I & Sedivy - chapter 4 & Thought Paper
Option A \\
3 & T 9/12 & Speech Perception II & Sedivy - chapter 7 & Thought Paper
Option B \\
3 & Th 9/14 & Words, Meaning, and Concepts & Sedivy - chapter 5 C\&C -
chapter 3 & Quiz 1 \\
4 & T 9/19 & Words, Meaning, and Concepts & Sedivy - chapter 8 & Thought
Paper Option C \\
4 & Th 9/21 & Words, Meaning, and Concepts & Chen \& Rogers (2014) & Lab
Report 1 \\
5 & T 9/26 & Sentences and Syntax & Sedivy - chapter 6 & Quiz 2 Thought
Paper Option D \\
5 & Th 9/28 & Sentence Processing & Sedivy - chapter 9 & \\
6 & T 10/3 & Speaking & Sedivy - chapter 10 & Quiz 3 Thought Paper
Option E \\
6 & Th 10/5 & Is language innate? & C\&C - chapter 4-5 Aslin \& Newport
(2012) Goldin-Meadow \& Mylander (1998) & \\
7 & T 10/10 & Is language special to humans? Non-human animal
communication & Sedivy - chapter 2 C\&C - chapter 7 Based on groups:
Herbranson (2012), Pepperberg (2002), or Ramus et al.~(2000) & Lab
Report 2 \\
7 & Th 10/12 & Is language special to humans? Language in machines &
C\&C - epilogue Kallens et al.~(2023) & Thought Paper Option F \\
8 & T 10/17 & First Language Acquisition / Language Development & Sedivy
- chapter 12 Based on groups: Goodluck (2010), Arunachalam \& Waxman
(2010), Gelman \& Meyer (2011), Romberg \& Saffran (2010), or Schwab \&
Lew-Williams (2016) & \\
8 & Th 10/19 & Second Language Acquisition & Juffs (2010) & Thought
Paper Option G \\
9 & T 10/24 & Reading I & Sedivy - chapter 11 Treiman (2000) & Thought
Paper Option H \\
9 & Th 10/26 & Reading II & Dehaene (2009) - selected pages & Lab Report
3 \\
10 & T 10/31 & Language and Communication Disorders I - Aphasia &
Doedens \& Meteyard (2020) & Quiz 4 Thought Paper Option I \\
10 & Th 11/2 & Language and Communication Disorders II - Dyslexia &
Ozernov-Palchik \& Gaab (2016) pages 152-162 & \\
11 & T 11/6 & Election Day; Sign Language and gesture & Brentari \&
Coppola (2013) Goldin-Meadow (2016) & Thought Paper Option J (last
option to choose what you are writing about!) \\
11 & Th 11/9 & Language and Culture & Sedivy - chapter 13 C\&C - chapter
6 & \\
12 & T 11/14 & Language and Thought & C\&C - chapter 8 Boroditsky (2001)
January \& Kako (2007) & Thought Paper Option K (if you haven't started
now -- you need to do the last 4!) \\
12 & Th 11/16 & Language in the Brain & Poeppel et al.~(2012) & Quiz
5 \\
13 & T 11/21 & Language in the Brain II & Hamilton \& Huth (2020) &
Thought Paper Option L (only 3 left!) \\
13 & Th 11/23 & Thanksgiving Break - No Classes & & \\
14 & T 11/28 & Language and Music & Gordon et al.~(2015) Jantzen (2017)
& Thought Paper Option M (only 2 left!) \\
14 & Th 11/30 & Presentations & & \\
15 & T 12/5 & Presentations & & Thought Paper Option N (only 1 left!) \\
15 & Th 12/6 & Last day of class, Presentations & & \\
Finals & T 12/12 & Reading Day & & \\
Finals & 12/13 - 12/19 & Finals & & \textbf{Final Paper} \\
\end{longtable}

\part{Assignments}

\hypertarget{lab-reports}{%
\chapter*{Lab Reports}\label{lab-reports}}
\addcontentsline{toc}{chapter}{Lab Reports}

\markboth{Lab Reports}{Lab Reports}

Over the course of the semester, you will complete three lab reports.
For each lab report, you will participate in online example experiments
of the Psychology of Language outside of class time. You may complete
these experiments using computer labs or your own computer. For each
assigned experiment, you are expected to read the accompanying article
(available on Canvas) to be able to describe the methodology,
hypotheses, and outcomes. Be sure to read the article after
participating in the experiment!

\hypertarget{lab-report-format}{%
\section*{Lab Report Format}\label{lab-report-format}}
\addcontentsline{toc}{section}{Lab Report Format}

\markright{Lab Report Format}

Your lab report should provide the following pieces of information:

\begin{enumerate}
\def\labelenumi{\arabic{enumi}.}
\tightlist
\item
  Name of the lab and the date of your participation.
\item
  Write a description of what you did during the experiment.
\item
  Identify and explain how the independent variable(s) was/were
  manipulated.
\item
  Identify and explain how the dependent variable(s) was/were measured.
\item
  State the experimental hypothesis.
\item
  State the outcomes of the experiment.
\item
  Describe whether you think your data are consistent with the outcomes
  of the experiment. Why or why not?
\item
  Suggest future directions, such as how the experiment might be
  modified to improve the investigation or examine effects in other
  populations.
\item
  Write the APA-formatted citation of the accompanying article.
\end{enumerate}

Write clearly, concisely, and with complete sentences. You should
complete the lab reports on your own. You should submit your lab reports
to Canvas via the assignment links on or before the due date/time to
receive full credit.

An example lab report can be found \href{SampleLabReport.pdf}{here}.

\hypertarget{lab-report-grading-rubric}{%
\section*{Lab Report Grading Rubric}\label{lab-report-grading-rubric}}
\addcontentsline{toc}{section}{Lab Report Grading Rubric}

\markright{Lab Report Grading Rubric}

\begin{longtable}[]{@{}
  >{\raggedright\arraybackslash}p{(\columnwidth - 4\tabcolsep) * \real{0.3333}}
  >{\raggedright\arraybackslash}p{(\columnwidth - 4\tabcolsep) * \real{0.3333}}
  >{\raggedright\arraybackslash}p{(\columnwidth - 4\tabcolsep) * \real{0.3333}}@{}}
\toprule\noalign{}
\begin{minipage}[b]{\linewidth}\raggedright
Criteria
\end{minipage} & \begin{minipage}[b]{\linewidth}\raggedright
Great Job
\end{minipage} & \begin{minipage}[b]{\linewidth}\raggedright
Needs Work
\end{minipage} \\
\midrule\noalign{}
\endhead
\bottomrule\noalign{}
\endlastfoot
1. Name and date & Provides accurate information (1 point) & Missing or
inaccurate (0 points) \\
2. Description & Provides clear and accurate description (1 point) &
Missing, unclear, or inaccurate (0 points) \\
3. Independent variables & Provides clear and accurate description (1
point) & Missing, unclear, or inaccurate (0 points) \\
4. Dependent variables & Provides clear and accurate description (1
point) & Missing, unclear, or inaccurate (0 points) \\
5. Experimental hypothesis & Provides clear and accurate description (1
point) & Missing, unclear, or inaccurate (0 points) \\
6. Outcomes & Provides clear and accurate description (1 point) &
Missing, unclear, or inaccurate (0 points) \\
7. Your data & Provides clear description and specifically discusses
relation between your experience and the outcomes written in the article
(1 point) & Missing, unclear, or does not discuss relation between your
experience and the outcomes written in the article (0 points) \\
8. Future directions & Provides unique future direction (1 point) & Does
not provide unique future direction (0 points) \\
9. APA-formatted citation & Provides accurate APA citation (1 point) &
Does not provide accurate APA citation (0 points) \\
Due date & Submitted on time (1 point) & Submitted late (0 points) \\
\end{longtable}

\hypertarget{lab-reports-this-semester}{%
\section*{Lab Reports this Semester}\label{lab-reports-this-semester}}
\addcontentsline{toc}{section}{Lab Reports this Semester}

\markright{Lab Reports this Semester}

\hypertarget{th-921-audio-visual-speech-in-noise-10-15-min}{%
\subsection*{Th 9/21, Audio-visual speech in noise (10-15
min)}\label{th-921-audio-visual-speech-in-noise-10-15-min}}
\addcontentsline{toc}{subsection}{Th 9/21, Audio-visual speech in noise
(10-15 min)}

\begin{itemize}
\tightlist
\item
  \href{https://research.sc/participant/login/dynamic/BB2C8E1A-D299-4456-AEB0-6BEB59C7FFF5}{Link
  to experiment}
\item
  Article: Karas et al.~(2019)
\end{itemize}

\hypertarget{th-928-sentence-prediction-5-min}{%
\subsection*{Th 9/28, Sentence prediction (5
min)}\label{th-928-sentence-prediction-5-min}}
\addcontentsline{toc}{subsection}{Th 9/28, Sentence prediction (5 min)}

\begin{itemize}
\tightlist
\item
  \href{https://research.sc/participant/login/dynamic/6824FDBF-4409-4B02-AE9E-10BA428B1D61}{Link
  to experiment}
\item
  Article: Gambi et al.~(2021)
\end{itemize}

\hypertarget{th-1026-learning-new-words-through-reading-15-20-min}{%
\subsection*{Th 10/26, Learning new words through reading (15-20
min)}\label{th-1026-learning-new-words-through-reading-15-20-min}}
\addcontentsline{toc}{subsection}{Th 10/26, Learning new words through
reading (15-20 min)}

\begin{itemize}
\tightlist
\item
  \href{https://research.sc/participant/login/dynamic/B3E7A61D-6F92-4B3A-B80A-8D8513D01C6B}{Link
  to experiment}
\item
  Article: Hulme et al.~(2022)
\end{itemize}

\hypertarget{thought-papers}{%
\chapter*{Thought Papers}\label{thought-papers}}
\addcontentsline{toc}{chapter}{Thought Papers}

\markboth{Thought Papers}{Thought Papers}

You must choose four options (one from each section) and complete those
papers by the due date. Each paper should be at least one page (not
including references) double spaced with 12-point font.

\hypertarget{section-1}{%
\section*{Section 1}\label{section-1}}
\addcontentsline{toc}{section}{Section 1}

\markright{Section 1}

\hypertarget{due-95-a-speech-perceptionintro}{%
\subsection*{Due 9/5, A: speech
perception/intro}\label{due-95-a-speech-perceptionintro}}
\addcontentsline{toc}{subsection}{Due 9/5, A: speech perception/intro}

In the Sedivy chapter, she writes about things that people say about
language that are almost certainly wrong (Table 1.1). Sedivy puts both
``You can learn language by watching television'' and ``You can't learn
language by watching television'' on this list. Doing your own research
online, write about the support for both claims. Do you think you can
learn language by watching television? Relevant readings: Sedivy chapter
1.

\hypertarget{due-912-b-speech-perception}{%
\subsection*{Due 9/12, B: speech
perception}\label{due-912-b-speech-perception}}
\addcontentsline{toc}{subsection}{Due 9/12, B: speech perception}

Can speech perception be improved through training and practice? If yes,
describe what might be a useful training paradigm. If no, explain why
not. Relevant readings: Sedivy chapters 4 and 7.

\hypertarget{due-919-c-conceptsspeech-production}{%
\subsection*{Due 9/19, C: concepts/speech
production}\label{due-919-c-conceptsspeech-production}}
\addcontentsline{toc}{subsection}{Due 9/19, C: concepts/speech
production}

How do emotions and stress impact speech production? Give at least three
examples of how different emotions influence speech production. Is
emotional speech easier or harder to understand than neutral speech?
Relevant readings: Sedivy chapters 5 and 10.

\hypertarget{due-926-d-concepts}{%
\subsection*{Due 9/26, D: concepts}\label{due-926-d-concepts}}
\addcontentsline{toc}{subsection}{Due 9/26, D: concepts}

How do our individual experiences and contexts shape our concepts? Can
you think of any examples of how your own experiences or cultural
background have influenced the way you understand certain concepts or
categories? Relevant readings: Sedivy chapter 8.

\hypertarget{section-2}{%
\section*{Section 2}\label{section-2}}
\addcontentsline{toc}{section}{Section 2}

\markright{Section 2}

\hypertarget{due-103-e-sentence-processing}{%
\subsection*{Due 10/3, E: sentence
processing}\label{due-103-e-sentence-processing}}
\addcontentsline{toc}{subsection}{Due 10/3, E: sentence processing}

Read the Introduction section (pages 1-2) of the
\href{papers/Kinreichetal2017.pdf}{Kinreich et al.~(2017)} paper. Why do
you think that couples would have more brain-to-brain synchrony than
strangers? Do you think this would apply only to romantic couples? Why
or why not? Why do you think our brains are capable of brain-to-brain
synchrony? In other words, what is the advantage of having this ability?
Relevant readings: Sedivy chapter 6 and 9.

\hypertarget{due-1010-f-is-language-innatespecial}{%
\subsection*{Due 10/10, F: Is language
innate/special?}\label{due-1010-f-is-language-innatespecial}}
\addcontentsline{toc}{subsection}{Due 10/10, F: Is language
innate/special?}

Based on the readings and your experience in class so far, do you think
that language is an innate human ability? Give a few examples that argue
for each side (language is innate and specific to humans vs.~language is
learned and is not specific to humans) and pick one at the end. Relevant
readings: C\&C chapters 4 and 5.

\hypertarget{due-1017-g-language-developmentlanguage-in-machines}{%
\subsection*{Due 10/17, G: Language development/language in
machines}\label{due-1017-g-language-developmentlanguage-in-machines}}
\addcontentsline{toc}{subsection}{Due 10/17, G: Language
development/language in machines}

Choose a few examples from the textbook of challenges that children face
in language development. Open \href{https://chat.openai.com/}{ChatGPT}
(requires Open AI account but is free),
\href{https://bard.google.com/}{Bard} (access through Google but is
free), or \href{https://www.bing.com/?scope=web\&FORM=HDRSC2}{Bing}
(requires Microsoft account but is free). Give prompts to these AI
language tools that match childrens' language challenges. Does the AI
tool respond similarly to children? Give the response and explain why
you think that the response matches or is different from children.
Relevant readings: C\&C chapter 7 and epilogue, Sedivy chapter 2.

\hypertarget{due-1024-h-second-language-acquisition}{%
\subsection*{Due 10/24, H: second language
acquisition}\label{due-1024-h-second-language-acquisition}}
\addcontentsline{toc}{subsection}{Due 10/24, H: second language
acquisition}

How does one's first language facilitate and hinder second language
acquisition? How does second language acquisition affect one's native
language? Give at least two examples of each case (facilitate, hinder,
affect). Relevant readings: Sedivy chapter 12.

\hypertarget{section-3}{%
\section*{Section 3}\label{section-3}}
\addcontentsline{toc}{section}{Section 3}

\markright{Section 3}

\hypertarget{due-1031-i-reading}{%
\subsection*{Due 10/31, I: Reading}\label{due-1031-i-reading}}
\addcontentsline{toc}{subsection}{Due 10/31, I: Reading}

How do digital technologies, such as e-books or online reading
platforms, affect the way that readers interact with and process written
texts? What are some of the advantages and disadvantages of these
technologies? Give a few examples of how you might design digital
reading environments to support optimal reading comprehension and
learning. Relevant readings: Dehaene chapters.

\hypertarget{due-117-j-disorders}{%
\subsection*{Due 11/7, J: Disorders}\label{due-117-j-disorders}}
\addcontentsline{toc}{subsection}{Due 11/7, J: Disorders}

How does dyslexia affect abilities outside of reading? Does having
dyslexia always make performance worse? Can you find an example of
something individuals with dyslexia are better or faster at than typical
individuals? Relevant readings: Dehaene chapters.

\hypertarget{due-1114-k-language-and-culturesign-language}{%
\subsection*{Due 11/14, K: Language and culture/sign
language}\label{due-1114-k-language-and-culturesign-language}}
\addcontentsline{toc}{subsection}{Due 11/14, K: Language and
culture/sign language}

How can signed language be used as a tool for promoting social justice
and equity for deaf and hard-of-hearing communities? Discuss the
potential impact of promoting signed language accessibility in society.
Relevant readings: C\&C chapter 6, Sedivy chapter 13.

\hypertarget{section-4}{%
\section*{Section 4}\label{section-4}}
\addcontentsline{toc}{section}{Section 4}

\markright{Section 4}

\hypertarget{due-1121-l-language-and-thoughtlanguage-in-the-brain}{%
\subsection*{Due 11/21, L: Language and thought/language in the
brain}\label{due-1121-l-language-and-thoughtlanguage-in-the-brain}}
\addcontentsline{toc}{subsection}{Due 11/21, L: Language and
thought/language in the brain}

Read the short article by \href{papers/Lupyanetal2007.pdf}{Lupyan et
al.~(2007)}. In your own words, describe the main questions tested in
the article and what the authors found. Why do you think that labels
help people learn faster? If you could change the design to this study
to improve learning even more, what do you think you would do? This can
involve manipulating the labels, the stimuli, or something else, but be
as specific as you can. Explain why you think this would help learning.

\hypertarget{due-1128-m-language-in-the-brain}{%
\subsection*{Due 11/28, M: Language in the
brain}\label{due-1128-m-language-in-the-brain}}
\addcontentsline{toc}{subsection}{Due 11/28, M: Language in the brain}

If you think about popular media about zombies, they vary in their motor
skills (some are slow, others are very fast) and sensory abilities (some
can't see, others have super hearing), but they usually don't vary in
their language skills. Zombies don't have language. Or do they? Using
what you know about language and the brain, why might zombies appear to
have language deficits? Give a reasonable explanation about how zombies
do not have the capacity for language and why this might be as well as
how zombies may have the capacity for language, but cannot express it
and why this might be. Relevant readings: Sedivy chapter 3.

\hypertarget{due-125-n-language-in-the-brain}{%
\subsection*{Due 12/5, N: Language in the
brain}\label{due-125-n-language-in-the-brain}}
\addcontentsline{toc}{subsection}{Due 12/5, N: Language in the brain}

Music, like language, involves a complex set of activities and mental
processes. Try to generate as detailed a list as you can of the various
components that go into musical and linguistic activity, going beyond
those discussed in the textbook. What makes someone good at music or
language? Once you've generated your lists, identify which of the skills
that are needed for music appear to have close analogues in language.
Where do you think it would be most likely that you'd see crossover in
cognitive processing? Create a proposal for how you might gather
evidence to support your idea of connections between music and language
skills. Relevant readings: Sedivy chapter 3.

\hypertarget{final-paper-and-presentation}{%
\chapter*{Final paper and
presentation}\label{final-paper-and-presentation}}
\addcontentsline{toc}{chapter}{Final paper and presentation}

\markboth{Final paper and presentation}{Final paper and presentation}

Forthcoming.



\end{document}
