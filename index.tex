% Options for packages loaded elsewhere
\PassOptionsToPackage{unicode}{hyperref}
\PassOptionsToPackage{hyphens}{url}
\PassOptionsToPackage{dvipsnames,svgnames,x11names}{xcolor}
%
\documentclass[
  letterpaper,
  DIV=11,
  numbers=noendperiod]{scrreprt}

\usepackage{amsmath,amssymb}
\usepackage{iftex}
\ifPDFTeX
  \usepackage[T1]{fontenc}
  \usepackage[utf8]{inputenc}
  \usepackage{textcomp} % provide euro and other symbols
\else % if luatex or xetex
  \usepackage{unicode-math}
  \defaultfontfeatures{Scale=MatchLowercase}
  \defaultfontfeatures[\rmfamily]{Ligatures=TeX,Scale=1}
\fi
\usepackage{lmodern}
\ifPDFTeX\else  
    % xetex/luatex font selection
\fi
% Use upquote if available, for straight quotes in verbatim environments
\IfFileExists{upquote.sty}{\usepackage{upquote}}{}
\IfFileExists{microtype.sty}{% use microtype if available
  \usepackage[]{microtype}
  \UseMicrotypeSet[protrusion]{basicmath} % disable protrusion for tt fonts
}{}
\makeatletter
\@ifundefined{KOMAClassName}{% if non-KOMA class
  \IfFileExists{parskip.sty}{%
    \usepackage{parskip}
  }{% else
    \setlength{\parindent}{0pt}
    \setlength{\parskip}{6pt plus 2pt minus 1pt}}
}{% if KOMA class
  \KOMAoptions{parskip=half}}
\makeatother
\usepackage{xcolor}
\setlength{\emergencystretch}{3em} % prevent overfull lines
\setcounter{secnumdepth}{5}
% Make \paragraph and \subparagraph free-standing
\ifx\paragraph\undefined\else
  \let\oldparagraph\paragraph
  \renewcommand{\paragraph}[1]{\oldparagraph{#1}\mbox{}}
\fi
\ifx\subparagraph\undefined\else
  \let\oldsubparagraph\subparagraph
  \renewcommand{\subparagraph}[1]{\oldsubparagraph{#1}\mbox{}}
\fi


\providecommand{\tightlist}{%
  \setlength{\itemsep}{0pt}\setlength{\parskip}{0pt}}\usepackage{longtable,booktabs,array}
\usepackage{calc} % for calculating minipage widths
% Correct order of tables after \paragraph or \subparagraph
\usepackage{etoolbox}
\makeatletter
\patchcmd\longtable{\par}{\if@noskipsec\mbox{}\fi\par}{}{}
\makeatother
% Allow footnotes in longtable head/foot
\IfFileExists{footnotehyper.sty}{\usepackage{footnotehyper}}{\usepackage{footnote}}
\makesavenoteenv{longtable}
\usepackage{graphicx}
\makeatletter
\def\maxwidth{\ifdim\Gin@nat@width>\linewidth\linewidth\else\Gin@nat@width\fi}
\def\maxheight{\ifdim\Gin@nat@height>\textheight\textheight\else\Gin@nat@height\fi}
\makeatother
% Scale images if necessary, so that they will not overflow the page
% margins by default, and it is still possible to overwrite the defaults
% using explicit options in \includegraphics[width, height, ...]{}
\setkeys{Gin}{width=\maxwidth,height=\maxheight,keepaspectratio}
% Set default figure placement to htbp
\makeatletter
\def\fps@figure{htbp}
\makeatother

\KOMAoption{captions}{tableheading}
\makeatletter
\@ifpackageloaded{tcolorbox}{}{\usepackage[skins,breakable]{tcolorbox}}
\@ifpackageloaded{fontawesome5}{}{\usepackage{fontawesome5}}
\definecolor{quarto-callout-color}{HTML}{909090}
\definecolor{quarto-callout-note-color}{HTML}{0758E5}
\definecolor{quarto-callout-important-color}{HTML}{CC1914}
\definecolor{quarto-callout-warning-color}{HTML}{EB9113}
\definecolor{quarto-callout-tip-color}{HTML}{00A047}
\definecolor{quarto-callout-caution-color}{HTML}{FC5300}
\definecolor{quarto-callout-color-frame}{HTML}{acacac}
\definecolor{quarto-callout-note-color-frame}{HTML}{4582ec}
\definecolor{quarto-callout-important-color-frame}{HTML}{d9534f}
\definecolor{quarto-callout-warning-color-frame}{HTML}{f0ad4e}
\definecolor{quarto-callout-tip-color-frame}{HTML}{02b875}
\definecolor{quarto-callout-caution-color-frame}{HTML}{fd7e14}
\makeatother
\makeatletter
\makeatother
\makeatletter
\@ifpackageloaded{bookmark}{}{\usepackage{bookmark}}
\makeatother
\makeatletter
\@ifpackageloaded{caption}{}{\usepackage{caption}}
\AtBeginDocument{%
\ifdefined\contentsname
  \renewcommand*\contentsname{Table of contents}
\else
  \newcommand\contentsname{Table of contents}
\fi
\ifdefined\listfigurename
  \renewcommand*\listfigurename{List of Figures}
\else
  \newcommand\listfigurename{List of Figures}
\fi
\ifdefined\listtablename
  \renewcommand*\listtablename{List of Tables}
\else
  \newcommand\listtablename{List of Tables}
\fi
\ifdefined\figurename
  \renewcommand*\figurename{Figure}
\else
  \newcommand\figurename{Figure}
\fi
\ifdefined\tablename
  \renewcommand*\tablename{Table}
\else
  \newcommand\tablename{Table}
\fi
}
\@ifpackageloaded{float}{}{\usepackage{float}}
\floatstyle{ruled}
\@ifundefined{c@chapter}{\newfloat{codelisting}{h}{lop}}{\newfloat{codelisting}{h}{lop}[chapter]}
\floatname{codelisting}{Listing}
\newcommand*\listoflistings{\listof{codelisting}{List of Listings}}
\makeatother
\makeatletter
\@ifpackageloaded{caption}{}{\usepackage{caption}}
\@ifpackageloaded{subcaption}{}{\usepackage{subcaption}}
\makeatother
\makeatletter
\@ifpackageloaded{tcolorbox}{}{\usepackage[skins,breakable]{tcolorbox}}
\makeatother
\makeatletter
\@ifundefined{shadecolor}{\definecolor{shadecolor}{rgb}{.97, .97, .97}}
\makeatother
\makeatletter
\makeatother
\makeatletter
\makeatother
\makeatletter
\@ifpackageloaded{fontawesome5}{}{\usepackage{fontawesome5}}
\makeatother
\ifLuaTeX
  \usepackage{selnolig}  % disable illegal ligatures
\fi
\IfFileExists{bookmark.sty}{\usepackage{bookmark}}{\usepackage{hyperref}}
\IfFileExists{xurl.sty}{\usepackage{xurl}}{} % add URL line breaks if available
\urlstyle{same} % disable monospaced font for URLs
\hypersetup{
  pdftitle={Psychology of Language},
  pdfauthor={Casey L. Roark},
  colorlinks=true,
  linkcolor={blue},
  filecolor={Maroon},
  citecolor={Blue},
  urlcolor={Blue},
  pdfcreator={LaTeX via pandoc}}

\title{Psychology of Language}
\author{Casey L. Roark}
\date{2023-08-01}

\begin{document}
\maketitle
\ifdefined\Shaded\renewenvironment{Shaded}{\begin{tcolorbox}[boxrule=0pt, interior hidden, borderline west={3pt}{0pt}{shadecolor}, enhanced, breakable, sharp corners, frame hidden]}{\end{tcolorbox}}\fi

\renewcommand*\contentsname{Table of contents}
{
\hypersetup{linkcolor=}
\setcounter{tocdepth}{2}
\tableofcontents
}
\bookmarksetup{startatroot}

\hypertarget{welcome}{%
\chapter*{Welcome!}\label{welcome}}
\addcontentsline{toc}{chapter}{Welcome!}

\markboth{Welcome!}{Welcome!}

\begin{figure}

{\centering \includegraphics[width=4.0625in,height=\textheight]{images/volodymyr-hryshchenko-V5vqWC9gyEU-unsplash.jpg}

}

\end{figure}

\hypertarget{learn-about-the-cognitive-and-neural-basis-of-our-remarkable-ability-to-communicate}{%
\subsubsection*{Learn about the cognitive and neural basis of our
remarkable ability to
communicate}\label{learn-about-the-cognitive-and-neural-basis-of-our-remarkable-ability-to-communicate}}
\addcontentsline{toc}{subsubsection}{Learn about the cognitive and
neural basis of our remarkable ability to communicate}

~

\hypertarget{psyc-712w---fall-2023}{%
\paragraph*{\texorpdfstring{{PSYC 712W - Fall
2023}}{PSYC 712W - Fall 2023}}\label{psyc-712w---fall-2023}}
\addcontentsline{toc}{paragraph}{{PSYC 712W - Fall 2023}}

\hypertarget{department-of-psychology}{%
\paragraph*{\texorpdfstring{{Department of
Psychology}}{Department of Psychology}}\label{department-of-psychology}}
\addcontentsline{toc}{paragraph}{{Department of Psychology}}

\hypertarget{university-of-new-hampshire}{%
\paragraph*{\texorpdfstring{{University of New
Hampshire}}{University of New Hampshire}}\label{university-of-new-hampshire}}
\addcontentsline{toc}{paragraph}{{University of New Hampshire}}

~

\hypertarget{instructor}{%
\subsection*{Instructor}\label{instructor}}
\addcontentsline{toc}{subsection}{Instructor}

\begin{itemize}
\tightlist
\item
  \faIcon{user} ~ \href{https://www.roarklab.com}{Dr.~Casey L. Roark}
\item
  \faIcon{university} ~ 440 McConnell Hall
\item
  \faIcon{envelope} ~
  \href{mailto:casey.roark@unh.edu}{\nolinkurl{casey.roark@unh.edu}}
\item
  \faIcon{twitter} ~
  \href{https://www.twitter.com/caseyroark}{caseyroark}
\item
  \faIcon{calendar-check} ~ Wednesday 1-3 pm or by appointment
\end{itemize}

\hypertarget{course-details}{%
\subsection*{Course details}\label{course-details}}
\addcontentsline{toc}{subsection}{Course details}

\begin{itemize}
\tightlist
\item
  \faIcon{calendar} ~ Tuesday and Thursday
\item
  \faIcon{calendar-alt} ~ Fall 2023
\item
  \faIcon{clock} ~ 2:10-3:30 pm
\item
  \faIcon{location-dot} ~ HORT 215
\end{itemize}

\hypertarget{contacting-me}{%
\subsection*{Contacting me}\label{contacting-me}}
\addcontentsline{toc}{subsection}{Contacting me}

E-mail is the best way to get in contact with me. I will try to respond
to all course-related e-mails within 24 hours (really), but also
remember that life can be busy and chaotic for everyone (including me!),
so if I don't respond right away, don't worry!

\bookmarksetup{startatroot}

\hypertarget{syllabus}{%
\chapter*{Syllabus}\label{syllabus}}
\addcontentsline{toc}{chapter}{Syllabus}

\markboth{Syllabus}{Syllabus}

\hypertarget{course-information}{%
\section*{Course information}\label{course-information}}
\addcontentsline{toc}{section}{Course information}

\markright{Course information}

\hypertarget{instructor-1}{%
\subsection*{Instructor}\label{instructor-1}}
\addcontentsline{toc}{subsection}{Instructor}

\begin{itemize}
\tightlist
\item
  \faIcon{user} ~ \href{www.roarklab.com}{Dr.~Casey L. Roark}
\item
  \faIcon{university} ~ 440 McConnell Hall
\item
  \faIcon{envelope} ~
  \href{mailto:casey.roark@unh.edu}{\nolinkurl{casey.roark@unh.edu}}
\item
  \faIcon{twitter} ~
  \href{https://www.twitter.com/caseyroark}{caseyroark}
\item
  \faIcon{calendar-check} ~ Wednesday 1-3 pm or by appointment
\end{itemize}

\hypertarget{course-details-1}{%
\subsection*{Course details}\label{course-details-1}}
\addcontentsline{toc}{subsection}{Course details}

\begin{itemize}
\tightlist
\item
  \faIcon{caret-right} ~ PSYC 712W -- Psychology of Language
\item
  \faIcon{check-square} ~ Pre-requisites: PSYC 402, 505, 512, 513 or
  permission
\item
  \faIcon{calendar} ~ Tuesday and Thursday
\item
  \faIcon{calendar-alt} ~ Fall 2023
\item
  \faIcon{clock} ~ 2:10-3:30 pm
\item
  \faIcon{location-dot} ~ HORT 215
\end{itemize}

\begin{tcolorbox}[enhanced jigsaw, toprule=.15mm, colbacktitle=quarto-callout-note-color!10!white, left=2mm, leftrule=.75mm, colframe=quarto-callout-note-color-frame, title=\textcolor{quarto-callout-note-color}{\faInfo}\hspace{0.5em}{Note}, arc=.35mm, bottomtitle=1mm, colback=white, bottomrule=.15mm, opacityback=0, coltitle=black, titlerule=0mm, rightrule=.15mm, opacitybacktitle=0.6, toptitle=1mm, breakable]

This syllabus is subject to change. Students will be promptly notified
of any changes.

\end{tcolorbox}

\hypertarget{course-overview}{%
\section*{Course Overview}\label{course-overview}}
\addcontentsline{toc}{section}{Course Overview}

\markright{Course Overview}

Psychology of Language explores the cognitive and neural bases of human
language. We use language in our everyday lives mostly effortlessly and
without thinking. But underneath it all, language is extraordinarily
complex. In this course, through lectures, reading, writing, and
discussion, we will explore some of the feats and challenges of human
language including (but not limited to):

\begin{itemize}
\tightlist
\item
  Components of language such as speech perception, speech production,
  reading, writing
\item
  Whether language is a uniquely human and innate ability
\item
  How children develop the ability to speak and understand language
\item
  How language functions in the brain
\item
  The relationship between language and thought
\item
  What happens when language does not function typically
\end{itemize}

\hypertarget{course-materials-required}{%
\section*{Course materials (required)}\label{course-materials-required}}
\addcontentsline{toc}{section}{Course materials (required)}

\markright{Course materials (required)}

\begin{itemize}
\tightlist
\item
  Christiansen, M. H. \& Chater, N. (2022).
  \href{https://bookshop.org/p/books/the-language-game-how-improvisation-created-language-and-changed-the-world-nick-chater/16984145?ean=9781541674981}{The
  Language Game: How improvisation created language and changed the
  world}. Basic Books.
\item
  Sedivy, J. (2014).
  \href{https://www.amazon.com/Language-Mind-Psycholinguistics-Julie-Sedivy/dp/1605357057}{Language
  in Mind: An introduction to psycholinguistics}. Second edition. Oxford
  University Press.
\item
  Other PDFs will be provided on the
  \href{https://my.unh.edu/canvas}{course website}.
\end{itemize}

\hypertarget{course-description}{%
\section*{Course Description}\label{course-description}}
\addcontentsline{toc}{section}{Course Description}

\markright{Course Description}

Theories of language structure, functions of human language, meaning,
relationship of language to other mental processes, language
acquisition, indices of language development, speech perception,
reading.

\hypertarget{course-learning-objectives}{%
\section*{Course Learning Objectives}\label{course-learning-objectives}}
\addcontentsline{toc}{section}{Course Learning Objectives}

\markright{Course Learning Objectives}

The goal in this course is to teach you the skills and concepts needed
to pose questions about the psychology of language in a rigorous and
scientific way, to research and evaluate scientific findings related to
these questions, and to critically evaluate and develop research
protocols that answer these questions. In the process, you will become
better able to evaluate scientific claims presented in the media and
elsewhere. You will participate actively in class lectures and
discussions. In addition, you will participate in exercises and
laboratories designed to introduce you to how to use science to
understand language. You will design a research project to gain a
hands-on understanding of the psychology of language. Whether you
continue in science or not, this course will benefit you in thinking
critically about everyday science claims in the media and in evaluating
scientific evidence for decision making. You will become a
critically-thinking consumer of science, learning to detect both flaws
and benefits of experimental designs and interpretations and to
formulate alternate explanations. If you complete the course
successfully, you will be able\ldots{}

\begin{itemize}
\tightlist
\item
  to recognize, analyze, and apply principles of scientific research
\item
  to discover what is known about a research topic
\item
  to understand how researchers study the psychology of language and
  when specific techniques are most useful
\item
  to evaluate different perspectives on fundamental questions about
  human language
\item
  to formulate a question about language in a way that can be tested
  empirically
\item
  to communicate research findings and your thoughts orally and in
  writing
\item
  to participate in experimental research and evaluate research findings
\item
  to employ critical thinking in evaluating scientific claims
\end{itemize}

\hypertarget{course-structure}{%
\section*{Course Structure}\label{course-structure}}
\addcontentsline{toc}{section}{Course Structure}

\markright{Course Structure}

\href{https://mycourses.unh.edu/courses}{myCourses} is the learning
management tool we use for this course. The course is organized by class
meetings. You will be able to find external readings and submit
assignments. Please do not message me through myCourses -- email me
instead.

\hypertarget{grades}{%
\section*{Grades}\label{grades}}
\addcontentsline{toc}{section}{Grades}

\markright{Grades}

\begin{longtable}[]{@{}
  >{\raggedright\arraybackslash}p{(\columnwidth - 4\tabcolsep) * \real{0.3261}}
  >{\raggedright\arraybackslash}p{(\columnwidth - 4\tabcolsep) * \real{0.1522}}
  >{\raggedright\arraybackslash}p{(\columnwidth - 4\tabcolsep) * \real{0.5217}}@{}}
\toprule\noalign{}
\begin{minipage}[b]{\linewidth}\raggedright
Item
\end{minipage} & \begin{minipage}[b]{\linewidth}\raggedright
\% of final grade
\end{minipage} & \begin{minipage}[b]{\linewidth}\raggedright
Requirements
\end{minipage} \\
\midrule\noalign{}
\endhead
\bottomrule\noalign{}
\endlastfoot
Quizzes & 15\% & There are 5 quizzes for this course, which are
scheduled on the course calendar. Each quiz is worth 3\% of the final
grade (5 quizzes x 3\% = 15\% total) \\
Class Attendance, Participation, and Discussion & 20\% & Attending and
participating in class are important parts of this course. Attendance
will be taken in each class period. Participation will be assessed for
each class period and students will be given multiple options for
participation over the course of the semester (e.g., group discussion,
pair discussion, self-reflection, in class activities, etc.) \\
Lab Reports & 15\% & Students will be required to complete three lab
reports during the semester. Each lab report will be worth 5\% of the
final grade (3 lab reports x 5\% = 15\% total) \\
Thought Papers & 20\% & There are four required thought papers for this
course. Each week will come with the opportunity to complete a thought
paper and students are expected to complete at least four over the
course of the semester. Each paper will be worth 5\% of the final grade
(4 papers x 5\% = 20\% total) \\
Final Presentation & 15\% & There will be a final presentation in the
final weeks of the semester. \\
Final Paper & 15\% & There will be a final paper along with the
presentation in the final weeks of the semester. \\
\end{longtable}

\hypertarget{course-policies}{%
\section*{Course Policies}\label{course-policies}}
\addcontentsline{toc}{section}{Course Policies}

\markright{Course Policies}

\hypertarget{office-hours}{%
\subsection*{Office hours}\label{office-hours}}
\addcontentsline{toc}{subsection}{Office hours}

Please watch this video on common misconceptions about office hours:

\url{https://vimeo.com/270014784?embedded=true\&source=vimeo_logo\&owner=2248721}

Office hours are set times dedicated to all of you. This means that I
will be in my office waiting for you to come by and talk to me with
whatever questions you have. This is the best and easiest way to find me
and the best chance for discussing class material and concerns. If you
are not available during my regularly scheduled office hours, please
email me to schedule a different time to meet.

\hypertarget{late-work}{%
\subsection*{Late work}\label{late-work}}
\addcontentsline{toc}{subsection}{Late work}

I would \textbf{highly recommend} staying caught up as much as possible,
but if you need to turn something in late, that's fine, just please let
me know at least one day in advance. Specific policies for late
assignments are listed for individual assignments.

\hypertarget{technical-requirements-and-technical-support}{%
\subsection*{Technical Requirements and Technical
Support}\label{technical-requirements-and-technical-support}}
\addcontentsline{toc}{subsection}{Technical Requirements and Technical
Support}

See \href{https://online.unh.edu/technical-requirements}{website
listings} for current recommendations and requirements related to this
course. For technical assistance please call (603) 862-4242 or fill out
an \href{https://itsupport.unh.edu/onlinelearning/}{online support
form}.

\hypertarget{university-disability-accommodations}{%
\subsection*{University Disability
Accommodations}\label{university-disability-accommodations}}
\addcontentsline{toc}{subsection}{University Disability Accommodations}

The University is committed to providing students with documented
disabilities equal access to all university programs and facilities. If
you think you have a disability requiring accommodations, you must
register with \href{http://www.unh.edu/studentaccessibility}{Student
Accessibility Services (SAS)} or directly contact SAS at (603)
862-2607.\\
Please provide me with that information privately so that we can review
those accommodations.

\hypertarget{academic-honesty}{%
\subsection*{Academic Honesty}\label{academic-honesty}}
\addcontentsline{toc}{subsection}{Academic Honesty}

Students are required to abide by the UNH Academic Honesty policy
located in the \href{https://catalog.unh.edu/srrr/}{Student Rights,
Rules, and Responsibilities Handbook}.

Plagiarism of any type may be grounds for receiving an ``F'' in an
assignment or an ``F'' in the overall course. Plagiarism is defined as
``the unattributed use of the ideas, evidence, or words of another
person, or the conveying the false impression that the arguments and
writing in a paper are your own.'' (UNH Academic Honesty Policy, 09.3)
Incidents are reported to the school dean and may be grounds for further
action. If you have questions about proper citation refer to your
department's writing guidelines. You can contact me at any time on this
issue. Additional resources can be found through the
\href{http://libraryguides.unh.edu/unhmcitingsources}{library guides on
citing sources}.

\hypertarget{expectations}{%
\subsection*{Expectations}\label{expectations}}
\addcontentsline{toc}{subsection}{Expectations}

The following guidelines will create a comfortable and productive
learning environment throughout the semester.

You can expect me to: - Respect you as individuals - Give timely
feedback - Assign work that meets the learning objectives of the course
- Adhere to the time expectations for a 4 credit course - Give
assessments that accurately reflect the material covered in class -
Provide clear and timely expectations for course requirements - Reply to
e-mails within 24-48 hours on weekdays

I expect you to: - Be attentive and engaged - Come to class ready to
participate (having done the readings) - Adhere to the highest level of
academic integrity - Spend an adequate amount of time studying course
materials each week so you can identify where you need clarification
early in the learning process - Seek help when appropriate - Communicate
with me about challenges

\hypertarget{general-well-being-and-stress}{%
\subsection*{General well-being and
stress}\label{general-well-being-and-stress}}
\addcontentsline{toc}{subsection}{General well-being and stress}

Do your best to follow guidelines to keep physically healthy. Do your
best to maintain a healthy lifestyle this semester by eating well,
exercising, avoiding drugs and alcohol, getting enough sleep and taking
some time to relax. This will help you achieve your goals and cope with
stress.

All of us benefit from support during times of struggle. You are not
alone. There are many helpful resources available on campus and an
important part of the college experience is learning how to ask for
help. Asking for support sooner rather than later is often helpful. It
is courageous to reach out for help.

If you or anyone you know experiences any academic stress, difficult
life events, or feelings like anxiety or depression, we strongly
encourage you to seek support.
\href{https://www.unh.edu/pacs/}{Psychological and Counseling Services
(PACS)} is here to help, (603) 862-2090. Consider reaching out to a
friend, faculty or family member you trust for help getting connected to
the support that can help. Your instructor can point you to resources.

If you or someone you know is feeling suicidal or in danger of
self-harm, call someone immediately, day or night you can contact Rapid
Response Access Point at 833-710-6477 or visit NH988.com. If the
situation is life threatening, call the police. (UNH Police: (603)
862-1427; Off campus: 911).

If you or someone you know is experiencing food insecurity -- worry
about affording food -- there are campus resources to help. Fill out a
\href{https://cm.maxient.com/reportingform.php?UnivofNH\&layout_id=15}{Swipe
it Forward request form} or visit the Cats' Cupboard in the MUB 140A.
See this
\href{https://www.unh.edu/dean-of-students/getting-help/housing-food-financial-basic-needs-support}{website}
for more Basic Needs Support and Resources.

\hypertarget{laurens-promise}{%
\subsection*{\texorpdfstring{\href{https://laurenmccluskey.org/laurens-promise/}{Lauren's
Promise}}{Lauren's Promise}}\label{laurens-promise}}
\addcontentsline{toc}{subsection}{\href{https://laurenmccluskey.org/laurens-promise/}{Lauren's
Promise}}

I will listen and believe you if someone is threatening you.

Lauren McCluskey, a 21-year-old honors student athlete,
\href{https://www.sltrib.com/opinion/commentary/2019/02/10/commentary-failing-lauren/}{was
murdered on October 22, 2018 by a man she briefly dated on the
University of Utah campus}. We must all take action to ensure that this
never happens again.

If you are in immediate danger, call 911 or UNH police ((603) 862-1427).

If you are experiencing sexual assault, domestic violence, or stalking,
please report it to me and I will connect you to resources or call UNH's
Psychological and Counseling Services ((603) 862-2090).

\bookmarksetup{startatroot}

\hypertarget{course-schedule}{%
\chapter*{Course Schedule}\label{course-schedule}}
\addcontentsline{toc}{chapter}{Course Schedule}

\markboth{Course Schedule}{Course Schedule}

\begin{longtable}[]{@{}
  >{\raggedright\arraybackslash}p{(\columnwidth - 8\tabcolsep) * \real{0.0886}}
  >{\raggedright\arraybackslash}p{(\columnwidth - 8\tabcolsep) * \real{0.1266}}
  >{\raggedright\arraybackslash}p{(\columnwidth - 8\tabcolsep) * \real{0.2405}}
  >{\raggedright\arraybackslash}p{(\columnwidth - 8\tabcolsep) * \real{0.2658}}
  >{\raggedright\arraybackslash}p{(\columnwidth - 8\tabcolsep) * \real{0.2785}}@{}}
\toprule\noalign{}
\begin{minipage}[b]{\linewidth}\raggedright
Week
\end{minipage} & \begin{minipage}[b]{\linewidth}\raggedright
Date
\end{minipage} & \begin{minipage}[b]{\linewidth}\raggedright
Topics
\end{minipage} & \begin{minipage}[b]{\linewidth}\raggedright
Readings
\end{minipage} & \begin{minipage}[b]{\linewidth}\raggedright
Assignments
\end{minipage} \\
\midrule\noalign{}
\endhead
\bottomrule\noalign{}
\endlastfoot
1 & T 8/29 & Introduction to Psychology of Language & & \\
1 & Th 8/31 & Science of Language & Sedivy - chapter 1 C\&C - chapters
1-2 & \\
2 & T 9/5 & Brief Introduction to Language in the Brain & Sedivy -
chapter 3 & \\
2 & Th 9/7 & Speech Perception I & Sedivy - chapter 4 & {Thought Paper
Option A} \\
3 & T 9/12 & Speech Perception II & Sedivy - chapter 7 & {Thought Paper
Option B} \\
3 & Th 9/14 & Words, Meaning, and Concepts I & C\&C - chapter 3 & Quiz
1 \\
4 & T 9/19 & Words, Meaning, and Concepts II & Sedivy - chapters 5 and 8
(up to 8.4) & {Thought Paper Option C} \\
4 & Th 9/21 & Sentences and Syntax & Sedivy - chapter 6 & Lab Report
1 \\
5 & T 9/26 & Sentence Processing & Sedivy - chapter 9 & {Thought Paper
Option D} \\
5 & Th 9/28 & Speaking & Sedivy - chapter 10 & Quiz 2 \\
6 & T 10/3 & Is language innate? & Sedivy - chapter 2 C\&C - chapter 4-5
& {Thought Paper Option E} \\
6 & Th 10/5 & Is language special to humans? Non-human animal
communication & C\&C - chapter 7 Based on groups: Berwick et al.~(2011),
Bray et al.~(2021), or Ramus et al.~(2000) & {Thought Paper Option F} \\
7 & T 10/10 & Is language special to humans? Language in machines & C\&C
- epilogue Contreras Kallens et al.~(2023) & Lab Report 2 \\
7 & Th 10/12 & First Language Acquisition / Language Development &
Sedivy - chapter 12 (up to 12.3) Schwab \& Lew-Williams (2016) & Quiz 3
{Thought Paper Option G} \\
8 & T 10/17 & Second Language Acquisition & Juffs (2010) & {Thought
Paper Option H} \\
8 & Th 10/19 & Reading & Sedivy - chapter 8 (starting 8.4) and 11 (up to
11.2) Treiman (2000) & Final paper proposal \\
9 & T 10/24 & Language and Communication Disorders I - Dyslexia &
Ozernov-Palchik \& Gaab (2016) pages 156-162 & {Thought Paper Option
I} \\
9 & Th 10/26 & Proposal feedback and time to work on project & & Lab
Report 3 \\
10 & T 10/31 & Language and Communication Disorders II - Aphasia &
Doedens \& Meteyard (2020) & Quiz 4 \\
10 & Th 11/2 & Sign Language and Gesture & Brentari \& Coppola (2013)
Goldin-Meadow (2016) & {Thought Paper Option J} \\
11 & T 11/7 & Election Day: Language and Culture & Sedivy - chapter 13
(up to 13.3) C\&C - chapter 6 & {Thought Paper Option K} \\
11 & Th 11/9 & Language and Thought & Sedivy - chapter 13 (starting
13.3)C\&C - chapter 8 & {Thought Paper Option L} \\
12 & T 11/14 & Language in the Brain I & Poeppel et al.~(2012) & \\
12 & Th 11/16 & Final paper draft peer-review (no class meeting) & &
Final paper draft \\
13 & T 11/21 & Language in the Brain II & Hamilton \& Huth (2020) & Quiz
5{Thought Paper Option M} \\
13 & Th 11/23 & Thanksgiving Break - No Classes & & \\
14 & T 11/28 & Language and Music & Gordon et al.~(2015) & {Thought
Paper Option N} \\
14 & Th 11/30 & Presentations & & \\
15 & T 12/5 & Presentations & & \\
15 & Th 12/7 & Last day of class, Presentations & & \\
Finals & T 12/12 & Reading Day & & \\
Finals & 12/13 - 12/19 & Finals & & \textbf{Final Paper} \\
\end{longtable}

\part{Assignments}

\hypertarget{lab-reports}{%
\chapter*{Lab Reports}\label{lab-reports}}
\addcontentsline{toc}{chapter}{Lab Reports}

\markboth{Lab Reports}{Lab Reports}

Over the course of the semester, you will complete three lab reports.
For each lab report, you will participate in online example experiments
of the Psychology of Language outside of class time. You may complete
these experiments using computer labs or your own computer. For each
assigned experiment, you are expected to read the accompanying article
(available on Canvas) to be able to describe the methodology,
hypotheses, and outcomes. Be sure to read the article after
participating in the experiment!

\hypertarget{lab-report-format}{%
\section*{Lab Report Format}\label{lab-report-format}}
\addcontentsline{toc}{section}{Lab Report Format}

\markright{Lab Report Format}

Your lab report should provide the following pieces of information:

\begin{enumerate}
\def\labelenumi{\arabic{enumi}.}
\tightlist
\item
  Name of the lab and the date of your participation.
\item
  Write a description of what you did during the experiment.
\item
  Identify and explain how the independent variable(s) was/were
  manipulated.
\item
  Identify and explain how the dependent variable(s) was/were measured.
\item
  State the experimental hypothesis.
\item
  State the outcomes of the experiment.
\item
  Describe whether you think your data are consistent with the outcomes
  of the experiment. Why or why not?
\item
  Suggest future directions, such as how the experiment might be
  modified to improve the investigation or examine effects in other
  populations.
\item
  Write the APA-formatted citation of the accompanying article.
\end{enumerate}

Write clearly, concisely, and with complete sentences. You should
complete the lab reports on your own. You should submit your lab reports
to Canvas via the assignment links on or before the due date/time to
receive full credit.

An example lab report can be found \href{SampleLabReport.pdf}{here}.

\hypertarget{lab-report-grading-rubric}{%
\section*{Lab Report Grading Rubric}\label{lab-report-grading-rubric}}
\addcontentsline{toc}{section}{Lab Report Grading Rubric}

\markright{Lab Report Grading Rubric}

\begin{longtable}[]{@{}
  >{\raggedright\arraybackslash}p{(\columnwidth - 4\tabcolsep) * \real{0.3333}}
  >{\raggedright\arraybackslash}p{(\columnwidth - 4\tabcolsep) * \real{0.3333}}
  >{\raggedright\arraybackslash}p{(\columnwidth - 4\tabcolsep) * \real{0.3333}}@{}}
\toprule\noalign{}
\begin{minipage}[b]{\linewidth}\raggedright
Criteria
\end{minipage} & \begin{minipage}[b]{\linewidth}\raggedright
Great Job
\end{minipage} & \begin{minipage}[b]{\linewidth}\raggedright
Needs Work
\end{minipage} \\
\midrule\noalign{}
\endhead
\bottomrule\noalign{}
\endlastfoot
Name and date & Provides accurate information (1 point) & Missing or
inaccurate (0 points) \\
Description & Provides clear and accurate description (1 point) &
Missing, unclear, or inaccurate (0 points) \\
Independent variables & Provides clear and accurate description (1
point) & Missing, unclear, or inaccurate (0 points) \\
Dependent variables & Provides clear and accurate description (1 point)
& Missing, unclear, or inaccurate (0 points) \\
Experimental hypothesis & Provides clear and accurate description (1
point) & Missing, unclear, or inaccurate (0 points) \\
Outcomes & Provides clear and accurate description (1 point) & Missing,
unclear, or inaccurate (0 points) \\
Your data & Provides clear description and specifically discusses
relation between your experience and the outcomes written in the article
(1 point) & Missing, unclear, or does not discuss relation between your
experience and the outcomes written in the article (0 points) \\
Future directions & Provides unique future direction (1 point) & Does
not provide unique future direction (0 points) \\
APA-formatted citation & Provides accurate APA citation (1 point) & Does
not provide accurate APA citation (0 points) \\
Due date & Submitted on time (1 point) & Submitted late (0 points) \\
\end{longtable}

\hypertarget{lab-reports-this-semester}{%
\section*{Lab Reports this Semester}\label{lab-reports-this-semester}}
\addcontentsline{toc}{section}{Lab Reports this Semester}

\markright{Lab Reports this Semester}

\hypertarget{th-921-audio-visual-speech-in-noise-10-15-min}{%
\subsection*{Th 9/21, Audio-visual speech in noise (10-15
min)}\label{th-921-audio-visual-speech-in-noise-10-15-min}}
\addcontentsline{toc}{subsection}{Th 9/21, Audio-visual speech in noise
(10-15 min)}

\begin{itemize}
\tightlist
\item
  \href{https://research.sc/participant/login/dynamic/BB2C8E1A-D299-4456-AEB0-6BEB59C7FFF5}{Link
  to experiment}
\item
  Article: Karas et al.~(2019)
\end{itemize}

\hypertarget{th-1010-word-learning-through-disfluent-speech-5-min}{%
\subsection*{Th 10/10, Word learning through disfluent speech (5
min)}\label{th-1010-word-learning-through-disfluent-speech-5-min}}
\addcontentsline{toc}{subsection}{Th 10/10, Word learning through
disfluent speech (5 min)}

\begin{itemize}
\tightlist
\item
  \href{https://research.sc/participant/login/dynamic/6824FDBF-4409-4B02-AE9E-10BA428B1D61}{Link
  to experiment}
\item
  Article: Libersky et al.~(2023)
\end{itemize}

\hypertarget{th-1026-learning-new-words-through-reading-15-20-min}{%
\subsection*{Th 10/26, Learning new words through reading (15-20
min)}\label{th-1026-learning-new-words-through-reading-15-20-min}}
\addcontentsline{toc}{subsection}{Th 10/26, Learning new words through
reading (15-20 min)}

\begin{itemize}
\tightlist
\item
  \href{https://research.sc/participant/login/dynamic/B3E7A61D-6F92-4B3A-B80A-8D8513D01C6B}{Link
  to experiment}
\item
  Article: Hulme et al.~(2022)
\end{itemize}

\hypertarget{thought-papers}{%
\chapter*{Thought Papers}\label{thought-papers}}
\addcontentsline{toc}{chapter}{Thought Papers}

\markboth{Thought Papers}{Thought Papers}

You must choose four options (one from each section) and complete those
papers by the due date. Each paper should be at least one page (not
including references) double spaced with 12-point font.

\hypertarget{section-1}{%
\section*{\texorpdfstring{{Section 1}}{Section 1}}\label{section-1}}
\addcontentsline{toc}{section}{{Section 1}}

\markright{{Section 1}}

\hypertarget{due-97-a-thinking-about-language}{%
\subsection*{Due 9/7, A: Thinking about
Language}\label{due-97-a-thinking-about-language}}
\addcontentsline{toc}{subsection}{Due 9/7, A: Thinking about Language}

In the Sedivy chapter, she writes about things that people say about
language that are almost certainly wrong (Table 1.1). Sedivy puts both
``You can learn language by watching television'' and ``You can't learn
language by watching television'' on this list. Doing your own research
online, write about the support for both claims. Do you think you can
learn language by watching television? Relevant readings: Sedivy chapter
1.

\hypertarget{due-912-b-speech-perception}{%
\subsection*{Due 9/12, B: Speech
Perception}\label{due-912-b-speech-perception}}
\addcontentsline{toc}{subsection}{Due 9/12, B: Speech Perception}

Can speech perception in noisy contexts be improved through training and
practice? If yes, describe what might be a useful training paradigm. If
no, explain why not. Relevant readings: Sedivy chapters 4 and 7.

\hypertarget{due-919-c-concepts}{%
\subsection*{Due 9/19, C: Concepts}\label{due-919-c-concepts}}
\addcontentsline{toc}{subsection}{Due 9/19, C: Concepts}

How do our individual experiences and contexts shape our concepts? Can
you think of any examples of how your own experiences or cultural
background have influenced the way you understand certain concepts or
categories? Relevant readings: Sedivy chapter 8.

\hypertarget{due-926-d-speech-production}{%
\subsection*{Due 9/26, D: Speech
Production}\label{due-926-d-speech-production}}
\addcontentsline{toc}{subsection}{Due 9/26, D: Speech Production}

How do emotions and stress impact speech production? Give at least three
examples of how different emotions influence speech production. Is
emotional speech easier or harder to understand than neutral speech?
Relevant readings: Sedivy chapter 9.

\hypertarget{section-2}{%
\section*{\texorpdfstring{{Section 2}}{Section 2}}\label{section-2}}
\addcontentsline{toc}{section}{{Section 2}}

\markright{{Section 2}}

\hypertarget{due-103-e-sentence-processing}{%
\subsection*{Due 10/3, E: Sentence
Processing}\label{due-103-e-sentence-processing}}
\addcontentsline{toc}{subsection}{Due 10/3, E: Sentence Processing}

Read the Introduction section (pages 1-2) of the article by
\href{papers/Kinreichetal2017.pdf}{Kinreich et al.~(2017)}. Why do you
think that couples would have more brain-to-brain synchrony than
strangers? Do you think this would apply only to romantic couples? Why
or why not? Why do you think our brains are capable of brain-to-brain
synchrony? In other words, what is the advantage of having this ability?
Relevant readings: Sedivy chapter 6 and 9.

\hypertarget{due-105-f-is-language-innatespecial}{%
\subsection*{Due 10/5, F: Is Language
Innate/Special?}\label{due-105-f-is-language-innatespecial}}
\addcontentsline{toc}{subsection}{Due 10/5, F: Is Language
Innate/Special?}

Based on the readings and your experience in class so far, do you think
that language is an innate human ability? Give a few examples that argue
for each side (language is innate and specific to humans vs.~language is
learned and is not specific to humans) and pick one at the end. Relevant
readings: C\&C chapters 4 and 5.

\hypertarget{due-1012-g-language-developmentlanguage-in-machines}{%
\subsection*{Due 10/12, G: Language Development/Language in
Machines}\label{due-1012-g-language-developmentlanguage-in-machines}}
\addcontentsline{toc}{subsection}{Due 10/12, G: Language
Development/Language in Machines}

Choose a few examples from the textbook of challenges that children face
in language development. Open \href{https://chat.openai.com/}{ChatGPT}
(requires Open AI account but is free),
\href{https://bard.google.com/}{Bard} (access through Google but is
free), or \href{https://www.bing.com/?scope=web\&FORM=HDRSC2}{Bing}
(requires Microsoft account but is free). Give prompts to these AI
language tools that match childrens' language challenges. Does the AI
tool respond similarly to children? Give the response and explain why
you think that the response matches or is different from children.
Relevant readings: C\&C chapter 7 and epilogue, Sedivy chapter 2.

\hypertarget{due-1017-h-second-language-acquisition}{%
\subsection*{Due 10/17, H: Second Language
Acquisition}\label{due-1017-h-second-language-acquisition}}
\addcontentsline{toc}{subsection}{Due 10/17, H: Second Language
Acquisition}

How does one's first language facilitate and hinder second language
acquisition? How does second language acquisition affect one's native
language? Give at least two examples of each case (facilitate, hinder,
affect). Relevant readings: Sedivy chapter 12.

\hypertarget{section-3}{%
\section*{\texorpdfstring{{Section 3}}{Section 3}}\label{section-3}}
\addcontentsline{toc}{section}{{Section 3}}

\markright{{Section 3}}

\hypertarget{due-1024-i-reading}{%
\subsection*{Due 10/24, I: Reading}\label{due-1024-i-reading}}
\addcontentsline{toc}{subsection}{Due 10/24, I: Reading}

How do digital technologies, such as e-books or online reading
platforms, affect the way that readers interact with and process written
texts? What are some of the advantages and disadvantages of these
technologies? Give a few examples of how you might design digital
reading environments to support optimal reading comprehension and
learning.

\hypertarget{due-112-j-language-and-communication-disorders}{%
\subsection*{Due 11/2, J: Language and Communication
Disorders}\label{due-112-j-language-and-communication-disorders}}
\addcontentsline{toc}{subsection}{Due 11/2, J: Language and
Communication Disorders}

How does dyslexia affect abilities outside of reading? Does having
dyslexia always make performance worse? Describe an example of something
individuals with dyslexia are better or faster at than typical
individuals and explain why this might be.

\hypertarget{due-116-k-language-and-culture}{%
\subsection*{Due 11/6, K: Language and
Culture}\label{due-116-k-language-and-culture}}
\addcontentsline{toc}{subsection}{Due 11/6, K: Language and Culture}

Read the Introduction section (pages 1-2) of the article by
\href{papers/Baileyetal2022.pdf}{Bailey et al.~(2022)}. In your own
words, describe the purpose and basic methodology of the study. More
generally, discuss how language shapes our perceptions of gender roles
and expectations.

\hypertarget{section-4}{%
\section*{\texorpdfstring{{Section 4}}{Section 4}}\label{section-4}}
\addcontentsline{toc}{section}{{Section 4}}

\markright{{Section 4}}

\hypertarget{due-1114-l-language-and-thought}{%
\subsection*{Due 11/14, L: Language and
Thought}\label{due-1114-l-language-and-thought}}
\addcontentsline{toc}{subsection}{Due 11/14, L: Language and Thought}

Read the short article by \href{papers/Lupyanetal2007.pdf}{Lupyan et
al.~(2007)}. In your own words, describe the main questions tested in
the article and what the authors found. Why do you think that labels
help people learn faster? If you could change the design to this study
to improve learning even more, what do you think you would do? This can
involve manipulating the labels, the stimuli, or something else, but be
as specific as you can. Explain why you think this would help learning.

\hypertarget{due-1121-m-language-in-the-brain}{%
\subsection*{Due 11/21, M: Language in the
Brain}\label{due-1121-m-language-in-the-brain}}
\addcontentsline{toc}{subsection}{Due 11/21, M: Language in the Brain}

If you think about popular media about zombies, they vary in their motor
skills (some are slow, others are very fast) and sensory abilities (some
can't see, others have super hearing), but they usually don't vary in
their language skills. Zombies don't have language. Or do they? Using
what you know about language and the brain, why might zombies appear to
have language deficits? Give a reasonable explanation about how zombies
do not have the capacity for language and why this might be as well as
how zombies may have the capacity for language, but cannot express it
and why this might be. Relevant readings: Sedivy chapter 3.

\hypertarget{due-1128-n-language-and-music}{%
\subsection*{Due 11/28, N: Language and
Music}\label{due-1128-n-language-and-music}}
\addcontentsline{toc}{subsection}{Due 11/28, N: Language and Music}

Music, like language, involves a complex set of activities and mental
processes. Try to generate as detailed a list as you can of the various
components that go into musical and linguistic activity, going beyond
those discussed in the textbook. What makes someone good at music or
language? Once you've generated your lists, identify which of the skills
that are needed for music appear to have close analogues in language.
Where do you think it would be most likely that you'd see crossover in
cognitive processing? Create a proposal for how you might gather
evidence to support your idea of connections between music and language
skills.

\hypertarget{thought-paper-grading-rubric}{%
\section*{Thought Paper Grading
Rubric}\label{thought-paper-grading-rubric}}
\addcontentsline{toc}{section}{Thought Paper Grading Rubric}

\markright{Thought Paper Grading Rubric}

\begin{longtable}[]{@{}
  >{\raggedright\arraybackslash}p{(\columnwidth - 4\tabcolsep) * \real{0.2500}}
  >{\raggedright\arraybackslash}p{(\columnwidth - 4\tabcolsep) * \real{0.5000}}
  >{\raggedright\arraybackslash}p{(\columnwidth - 4\tabcolsep) * \real{0.2500}}@{}}
\toprule\noalign{}
\begin{minipage}[b]{\linewidth}\raggedright
Criteria
\end{minipage} & \begin{minipage}[b]{\linewidth}\raggedright
Great Job
\end{minipage} & \begin{minipage}[b]{\linewidth}\raggedright
Needs Work
\end{minipage} \\
\midrule\noalign{}
\endhead
\bottomrule\noalign{}
\endlastfoot
Name and date & Provides accurate information (1 point) & Missing or
inaccurate (0 points) \\
Length & At least one page not including references (1 point) & Less
than required (0 points) \\
Formatting & Double spaced, 12 point font (1 point) & Inaccurate
formatting (0 points) \\
Content & Provides clear, thoughtful discussion of the prompt that shows
the student read the paper (if required by prompt) and meaningfully
engaged in thinking about the prompt and their response (6 points) &
Unclear or does not demonstrate meaningful engagement with the prompt
(0-5 points) \\
Due date & Submitted on time (1 point) & Submitted late (0 points) \\
\end{longtable}

\hypertarget{final-paper-and-presentation}{%
\chapter*{Final paper and
presentation}\label{final-paper-and-presentation}}
\addcontentsline{toc}{chapter}{Final paper and presentation}

\markboth{Final paper and presentation}{Final paper and presentation}

The goal of the final paper and presentation is for you to think deeply
about how we use experimental methods to answer questions about the
psychology of language. You will write a paper and give a presentation
that outlines a research proposal for a new experimental study about the
psychology of language. You are welcome to design a follow up study to
one of the studies we read about in class or choose a different topic
that interests you.

There are five milestones for this assignment:

\begin{enumerate}
\def\labelenumi{\arabic{enumi}.}
\item
  \textbf{By 10/19 at 2:10PM}, submit a short proposal (1/2-1 page) of a
  specific research question you will explore in the final paper and
  presentation. The proposal should detail some questions that you are
  interested in investigating. It should also include at least two
  empirical papers (outside of course materials) that you plan to read.
  I highly suggest you spend some time before this deadline beginning to
  think about this project so I can give you specific feedback and get
  you on the right track from the beginning. You can submit one or two
  ideas for feedback.
\item
  \textbf{By 11/15 at 9PM}, complete a detailed outline or rough draft
  (\textasciitilde3-6 pages) of the final paper that will be
  peer-reviewed by a classmate and reviewed by me. This outline/draft
  should include the research question and any sub-questions, a detailed
  overview of the experimental method that you would use to answer the
  research question, anticipated/hypothesized results. You should cite
  at least five empirical sources (outside of course materials).
\item
  \textbf{By 11/16 at 9PM} (especially during the class period on
  11/16), peer-edit the assigned paper. Read the draft using track
  changes in Microsoft Word to make any suggestions or comments. Write
  down any question you have -- were any parts confusing? Write down
  parts that you liked. It is possible that these edits will not be
  anonymized, so keep that in mind. Submit the edited draft to Canvas.
\item
  \textbf{By 11/30} (Th 11/30, T 12/5, Th 12/7), you should prepare a
  5-7 minute presentation about your proposed project. You should
  include an introduction that lays out the purpose/question of the
  planned experiment, the method of the planned experiment, and the
  anticipated/hypothesized results. You should be prepared to answer
  questions about your presentation.
\item
  By \textbf{TBD final date}, you should submit a final draft of the
  paper, incorporating feedback from the outline/draft. The paper should
  be at least 1500 words long (\textasciitilde{} 6 double spaced pages
  in 12 pt font, not including references or title page). The paper
  should be in APA format.
\end{enumerate}

The paper should have the following parts:

\begin{itemize}
\tightlist
\item
  Title (on its own page)
\item
  Abstract (on same pages as other sections below)
\item
  Background and significance
\item
  Proposed experiment(s)

  \begin{itemize}
  \tightlist
  \item
    Hypothesis/hypotheses
  \item
    Experimental design
  \item
    Possible anticipated/hypothesized results
  \item
    Limitations
  \end{itemize}
\item
  References
\end{itemize}

\hypertarget{final-paper-grading-rubric}{%
\section*{Final Paper Grading Rubric}\label{final-paper-grading-rubric}}
\addcontentsline{toc}{section}{Final Paper Grading Rubric}

\markright{Final Paper Grading Rubric}

\begin{longtable}[]{@{}
  >{\raggedright\arraybackslash}p{(\columnwidth - 4\tabcolsep) * \real{0.2500}}
  >{\raggedright\arraybackslash}p{(\columnwidth - 4\tabcolsep) * \real{0.5000}}
  >{\raggedright\arraybackslash}p{(\columnwidth - 4\tabcolsep) * \real{0.2500}}@{}}
\toprule\noalign{}
\begin{minipage}[b]{\linewidth}\raggedright
Criteria
\end{minipage} & \begin{minipage}[b]{\linewidth}\raggedright
Great Job
\end{minipage} & \begin{minipage}[b]{\linewidth}\raggedright
Needs Work
\end{minipage} \\
\midrule\noalign{}
\endhead
\bottomrule\noalign{}
\endlastfoot
Proposal & Submitted on time (1 point) & Submitted late (0 points) \\
Outline/Rough Draft & Submitted on time (1 point) & Submitted late (0
points) \\
Peer-edited Draft & Submitted on time (3 points) & Submitted late (0
points) \\
Final Draft & Submitted on time (1 points) & Submitted late (0
points) \\
Title Page & Present in APA format (1 point) & Missing or inaccurate (0
points) \\
Abstract & 150-250 words that clearly describes importance of topic (1
point), introduces proposed experiment (1 point), gives basic
experimental design information (1 point), provides summary of
anticipated results (1 point), and big-picture conclusion about topic (1
point) & Missing elements (-1 point for each) \\
Background and significance & At least five paragraphs (1 point) that
introduces topic (2 points), reviews at least 5 articles from the
literature (5 points), and provides context for topic and hypothesis (2
points) & Missing or unclear elements (-1 point for each) \\
Proposed experiment & Gives detailed description of hypothesis (1
point), experimental design (2 points), anticipated results (2 points),
and limitations (1 point) & Missing or unclear elements (-1 point for
each) \\
References & APA formatted reference list of all in-text citations (1
point) & Missing or inaccurate (0 points) \\
Length and formatting & 1500 words, 6 pages (without title page or
references), double spaced in 12 point font (1 point) & Less than
required (0 points) \\
\end{longtable}

\hypertarget{final-presentation-grading-rubric}{%
\section*{Final Presentation Grading
Rubric}\label{final-presentation-grading-rubric}}
\addcontentsline{toc}{section}{Final Presentation Grading Rubric}

\markright{Final Presentation Grading Rubric}

\begin{longtable}[]{@{}
  >{\raggedright\arraybackslash}p{(\columnwidth - 4\tabcolsep) * \real{0.2500}}
  >{\raggedright\arraybackslash}p{(\columnwidth - 4\tabcolsep) * \real{0.5000}}
  >{\raggedright\arraybackslash}p{(\columnwidth - 4\tabcolsep) * \real{0.2500}}@{}}
\toprule\noalign{}
\begin{minipage}[b]{\linewidth}\raggedright
Criteria
\end{minipage} & \begin{minipage}[b]{\linewidth}\raggedright
Great Job
\end{minipage} & \begin{minipage}[b]{\linewidth}\raggedright
Needs Work
\end{minipage} \\
\midrule\noalign{}
\endhead
\bottomrule\noalign{}
\endlastfoot
Due date & Present on assigned date (5 points) & Present late (0
points) \\
Timing & 5-7 minutes (5 points) & Under 5 minutes or over 7 minutes (-\#
points depending on how much) \\
Introduce topic & Provides clear and accurate description (5 points) &
Missing, unclear, or inaccurate (-\# points depending on how much) \\
Experimental method & Provides clear and accurate description of
experimental methods including explicit definition of independent
variables (1 point), dependent variables (1 point), experimental
hypothesis (2 point), and an example from the task (2 points) & Missing,
unclear, or inaccurate (-1 or 2 for each) \\
Anticipated results & Provides clear and logical description of
anticipated results (with clear dependent variable units: 2 points) and
how this relates to experimental hypothesis (2 point), as well as at
least one alternative hypothesis (1 point) & Missing or unclear (-1 or 2
for each) \\
Limitations and future directions & Provides at least one limitation of
the experimental design (2 points) and one future direction (2 points) &
Missing or unclear (-2 for each) \\
\end{longtable}



\end{document}
